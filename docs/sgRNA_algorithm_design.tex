\documentclass{article}[12pt]
\usepackage{amsmath}
\usepackage{natbib}
\usepackage{graphicx}
\usepackage{bbm}
\usepackage{hyperref}

\usepackage[top=1in, bottom=1.25in, left=1.25in, right=1.25in]{geometry}


\usepackage{setspace}
\linespread{1.25}


\date{\today}

\author{Timothy Daley}

\title{Finding possible sgRNAs in a specified regions}

\begin{document}
\maketitle

Suppose that we have a list of genes and we wish to
design short guide RNAs (sgRNAs) for CRISPR knockout,
activation, or inhibition (ko/a/i).  The regions for each type are 
different.  For CRISPRko we typically want sgRNAs that are
located in first exon of the transcript; in CRISPRa we want to
target the promoter region of the gene, typically 400 to 50
bases upstream of the transcription start site (TSS); and
for CRISPRi we want to target around the TSS and 
slightly downstream (e.g. -50 to +300 from the TSS).

We will split up the design into the following steps:
\begin{enumerate}
\item From the gene names, obtain the sequences of the 
targeted regions.  
\item Given the DNA sequence of the targeted region,
find sgRNA's that are acceptable by a given set of rules.
\end{enumerate}

Step 1 can be done using R and biomaRt 
(\url{https://bioconductor.org/packages/release/bioc/html/biomaRt.html})
if the rules for finding the target region are specified.

Step 2 is where the meat of the problem lies.
Typically one wishes to find guide sequences that are sufficiently
different from all other similar length sequences in the genome.
Current tools look for sequences that are at least a prespecified
edit distance (e.g. $> 2$) using existing tools.  
This ignores the varying importance of positions on the guide RNA.
For example, the PAM distal bases are known to be
most critical to binding specificity.  Accordingly these bases are
known as the seed region are defined as between 7 and 12 base
pairs closest to the PAM.  
If all other subsequences of the genome are at least one edit distance away
from the seed region, then binding will occur with few off-target effects.
Ignoring this can reduce the set of possible target sequences.
Our goal here is to find the set of sgRNAs within the
input sequences with either no other sequences matching the
seed region or at least 2 mismatches from the total sequence
in a high throughput manner so that we can handle a large number of
target sequences and a large reference genome.

Given input:
\begin{enumerate}
\item DNA sequences of target region in FASTA format,
\item Reference genome in FASTA format,
\end{enumerate}
we do the following:
\begin{enumerate}
\item Identify all possible sgRNAs in the target regions using 
PAM,
\item Construct a hash table using seed regions of possible sgRNAs,
\item Hash the genome to find matches to the seed region, if there's a 
match then calculate edit distance, only one match means
no subsequences in the genome match the seed region.
\end{enumerate}

\section*{Constructing the hash table}


We will use the seed regions of the possible sgRNAs
as the hash values.  To make hashing efficient, we
want to iteratively hash the genome using the Rabin-Karp.  

Let $S$ denote the seed region.  For each
letter $S_{i}, i = 1, \ldots, |S|$ in the seed region
we can map each base to a unique integer
\[
\phi(S_{i}) = \begin{cases}
0 & \text{ if } S_{i} = \text{A} \\
1 & \text{ if } S_{i} = \text{C} \\
2 & \text{ if } S_{i} = \text{G} \\
3 & \text{ if } S_{i} = \text{T}.
\end{cases}
\]
Each unique nucleotide sequence of length $|S|$
can be represented as a unique number
\begin{equation} \label{string2num}
\phi(S) = \sum_{i = 1}^{|S|} 4^{i-1} \cdot \phi(S_{i})
\end{equation}
that is between $0$ and $4^{|S|}$.  This number
will be the hash value of the seed sequence.

\subsection*{Rabin-Karp shifting}

Suppose we are traversing the genome $G$ iteratively.
Given a position $j$ in the genome with sequence
$G_{j} G_{j+1} \cdots G_{j + |S| - 1}$, the corresponding
hash value of this sequence is 
\[
\phi(G_{j} G_{j+1} \cdots G_{j + |S| - 1}) = \sum_{i = 0}^{|S| - 1} 4^{i} \cdot \phi(G_{j + i}).
\]
The next position in the genome, $j + 1$ has hash value 
\[
\phi(G_{j + 1} G_{j+2} \cdots G_{j + |S|} ) = \sum_{i = 0}^{|S| -1} 4^{i} \cdot \phi(G_{j + 1 + i}).
\]

By definition, the last $|S| - 1$ bases of the first sequence
are the first $|S| - 1$ bases of the second sequence.
Therefore we can obtain $\phi(G_{j + 1} G_{j+2} \cdots G_{j + |S|} )$
from $\phi(G_{j} G_{j+1} \cdots G_{j + |S| - 1}) $ by subtracting the 
first base in the sequence, shifting the numerical base, and
then adding the last base.  
More simply,
\begin{align}
\phi(G_{j + 1} G_{j+2} \cdots G_{j + |S|} ) &= 
4^{|S| - 1} \phi( G_{ j + |S|} ) + \sum_{i = 1}^{|S| - 1} 4^{i - 1} \cdot \phi( G_{j + i} )
\notag \\
&= 4^{|S| - 1} \phi( G_{ j + |S|} ) + 4^{-1} \sum_{i = 1}^{|S| - 1} 4^{i} \cdot \phi (G_{j + i}) 
\notag \\
&= 4^{|S| - 1} \phi( G_{ j + |S|} ) + 4^{-1} \big( \sum_{i = 0}^{|S| - 1} 4^{i} \cdot \phi(G_{j + i} )
- \phi(G_{j}) \big)
\notag \\
&= 4^{|S| - 1} \phi( G_{ j + |S|} ) + 4^{-1} \big( \phi(G_{j} G_{j+1} \cdots G_{j + |S| - 1}) - \phi(G_{j}) \big)
\notag
\end{align}



\end{document}